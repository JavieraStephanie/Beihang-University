% !Mode:: "TeX:UTF-8"
% Author: Zhengxi Tian
% Email: zhengxi.tian@hotmail.com

\chapter*{结论\markboth{结论}{}}
\addcontentsline{toc}{chapter}{结论}

本文提出了一个基于注意力机制和双向循环GRU的框架,用以对文本的情感极性进行预测和分类。
与以往的实验不同,本文提出的框架主要的创新点在于,采用深度学习的方法,从“句子序列建模”和“单词特征捕捉”两个层面
对网络上的评论文本进行情感分析。

本文首先介绍了情感分析领域的基本研究方法,包含从传统的机器学习方法到当下流行的深度学习方法。
具体来看,本文介绍了RNN在情感分析任务中的独特优势在于能够在时间间隔内学习单词之间的基本关系。
但是,随之带来的梯度消失和爆炸问题也成为一大限制,所以LSTM和GRU等高级的神经网络开始应用于该领域。
接下来,本文分析了GRU网络的优点,并且讨论了句子级别的深度学习和语法依赖在情感分析中的重要性。
此外,本文还强调了需要关注句子中的特殊情感词汇,从单词特征级别对输入序列的关键信息进行提取。

所以,本文的框架首先设计了一个预注意力双向循环GRU,该部分可以从句子两端起始分别进行单词流的读入。
这样通过双向的句子识别,可以对句子进行一个整体的建模,有助于框架完整地理解句子的基本语义依赖,和单词之间的潜在关系。
采用双向循环GRU从“句子序列”级别建模,可以有效地建立起情感感知体系,有助于下一步对文本情感的预测。
接下来,本文从“单词特征捕捉”的角度出发,引入注意力机制对输入序列中的特定情绪词汇进行特征捕捉,
注意力机制不仅可以对句子中的关键信息进行捕捉和分析,还能在较长的序列中有良好表现。
之后,本文模仿解码器的功能,在框架最后搭建了一个后注意力单向GRU,来进一步提取预测的情感特征。
这一设计的初衷是模仿人类的阅读习惯,在阅读较长较复杂的文本时,会对语句进行重复阅读来保证情感分类的准确性。
最后,本文将后注意力单向GRU的最后一层隐藏单元输入到softmax分类器中,得到最终的情感预测结果。

本文在四个标准的数据集上对框架的预测准确率和性能进行了评测,和基准线的比较可以发现本文提出的框架在情感分析中具有较强的优势。
虽然本文提出的框架在情感分类的准确性上有一定的优势,但在未来的工作中,还可以从以下几个方面进一步开展研究:

1)本文计划将框架再在其他的标准数据集上进行进一步的测试和分析,从而全面地证明框架的能力。

2)本文计划进一步完善预注意力双向循环GRU部分,比如尝试采用CNN和GRU或LSTM相结合的方式,通过构建更复杂高级的网络结构来对情感进行更准确的预测。

3)本文不仅希望可以对文本的情感进行二分类的预测,还希望可以从Aspect-level角度,对句子中的不同对象进行多分类的情感预测。
